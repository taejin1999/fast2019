\section{Conclusions}

We have presented a new stream management technique, \textsf{\small PCStream},
for multi-streamed SSDs.  Unlike existing stream management techniques,
\textsf{\small PCStream} fully automates the process of mapping data to a
stream based on PCs, which work well for append-only workloads as well as
in-place update workloads.  By exploiting an observation that most PCs are
distinguishable from each other in their lifetime characteristics,
\textsf{\small PCStream} allocates each PC to a different stream.  When a PC
has a large variance in their lifetimes, \textsf{\small PCStream} refines its
stream allocation during garbage collection and moves the long-lived data of
the current stream to the corresponding internal stream.  Our experimental
results show that \textsf{\small PCStream} can reduce the average WAF by up to
69\% over the existing automatic technique.

The current version of \textsf{\small PCStream} can be extended in several
directions.  First, the current version of PCStream does not support
applications with a write buffer ({\it e.g.,} MySQL). To address this, we plan
to extend PCStream interfaces so that application developers can easily
incorporate PCStream into their write buffering modules with minimal efforts.
Second, we have only considered \texttt{write()}-related systems calls to
collect PCs, but many applications ({\it e.g.} MongoDB) heavily use another
interface to access files, \texttt{mmap()}. In the future, we plan to extend
PCStream to support \texttt{mmap()} and evaluate its effectiveness using
various applications.
