\vspace{-10pt}
\section{Related Work}
\vspace{-5pt}

There have been many studies for multi-streamed SSDs ~\cite{MultiStream, Level,
vStream, FStream, AutoStream, PCStream}.  Kang {\it et al.} first proposed a
multi-streamed SSD that supported manual stream allocation for separating
different types of data~\cite{MultiStream}.  Yang {\it et al.} showed that a
multi-streamed SSD was effective for separating data of append-only
applications like RocksDB~\cite{Level}.  Yong {\it et al.} presented a virtual
stream management technique that resolved a limited number of streams.  While
above studies involve modifying the source code of target programs, the
stream allocation in PCStream is done automatically without any code
modification.

Rho {\it et al.} proposed a stream management technique, called FStream, at the
file system layer~\cite{FStream}. In FStream, metadata, journal
data, and user data that may have different lifetime characteristics were
allocated to separate streams.  Since FStream was implemented as part of a file
system, it was not able to directly detect application's I/O behaviors.
Also, it may be hard to be deployed in practice due to 
dependency with specific file-system implementations. PCStream is not
only able to detect application-specific behaviors, but also does not require
any modification of file systems.

Yang {\it et al.} presented the automatic stream management technique at the
block device layer. Similar to hot-cold separation employed in FTLs,
it decided the lifetime of data based on update frequencies of LBAs.
It didn't work well for applications ({\it e.g.}, RocksDB
and GCC) which have no a direct correlation between hotness of data
and LBAs.  PCStream detects the lifetime of data using PCs, so it
performs well even for append-only and write-once workloads.

Ha {\it et al.} proposed an idea of using PCs to separate hot data from cold
one in an FTL layer~\cite{PCHa}.  Kim {\it et al.} extended it for
multi-streamed SSDs~\cite{PCStream}.  Our work has improved the above
studies in a more complete fashion by showing 1) the important of the
internal stream guided by a host-level stream manager; 2) the effectiveness of
PCs for various workloads, including append-only, in-place update, and
write-once patterns; and 3) the unique and consistent nature of PCs which makes
it possible to capture long-term I/O behaviors of even short-lived
applications.


