\vspace{-10pt}
\section{Related Work}
\vspace{-5pt}

There have been many studies for multi-streamed SSDs ~\cite{MultiStream, Level,
vStream, FStream, AutoStream, PCStream}.  Kang {\it et al.} first proposed a
multi-streamed SSD that supported manual stream allocation for separating
different types of data~\cite{MultiStream}.  Yang {\it et al.} showed that a
multi-streamed SSD was effective for separating data of append-only
applications like RocksDB~\cite{Level}.  Yong {\it et al.} presented a virtual
stream management technique that allows logical streams, not physical streams,
to be allocated by applications.
Unlike these studies that involve modifying the source code of target programs, 
\textsf{\small PCStream} automates the stream allocation with no manual code modification.

\begin{comment}
Rho {\it et al.} proposed a stream management technique, called FStream, at the
file system layer~\cite{FStream}. In FStream, metadata, journal
data, and user data that may have different lifetime characteristics were
allocated to separate streams.  Since FStream was implemented as a part of a file
system, it was not able to directly detect application's I/O behaviors.
Also, it may be hard to be deployed in practice due to 
a strong dependence on file system-specific implementation details.
\textsf{\small PCStream}, on the other hand, efficiently exploits
programs' I/O behaviors using PCs with no file system-specific modifications.
\end{comment}

Yang {\it et al.} presented an automatic stream management technique at the
block device layer~\cite{AutoStream}. Similar to hot-cold data separation technique used in FTLs,
it approximates the data lifetime of data based on update frequencies of LBAs.
The applicability of this technique is, however, quite limited to in-place update workloads only.  
\textsf{\small PCStream} has no such limitation on the workload characteristics, thus effectively working for general
I/O workloads including append-only workloads, write-once workloads as well as in-place update workloads.

Ha {\it et al.} proposed an idea of using PCs to separate hot data from cold
one in an FTL layer~\cite{PCHa}.  Kim {\it et al.} extended it for
multi-streamed SSDs~\cite{PCStream}.  
Unlike these work, our study treats the PC-based stream management problem in a more complete fashion by 
showing 1) pinpointing the key weaknesses of existing multi-streamed SSD solutions, 
2) extending the effectiveness of PCs for more general I/O workloads 
including write-once patterns, and 3)
introducing internal streams as an effective solution for outlier PCs.  
Furthermore, \textsf{\small PCStream} exploits the globally unique nature of a PC signature 
for supporting short-lived applications.

