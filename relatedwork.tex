\section{Related Work}
There have been many studies that exploit multi-streamed SSDs to reduce
WAF~\cite{MultiStream, Level, vStream, FStream, AutoStream, PCStream}.  Kang
{\it et al.} first proposed a multi-streamed SSD that supported manual stream
allocation for separating different types of data~\cite{MultiStream}.  Yang
{\it et al.} showed that by manually annotating source code based on prior
knowledge about behaviors of programs, it would be possible to isolate
different types of data into separate streams, thereby lowering overall WAF and
improving performance~\cite{Level}.  Yong {\it et al.} presented a virtual
stream management technique that resolved limited number of streams.  All those
studies involve modifying the source code of target applications, leading to
increase deployment efforts.  In contrast to them, PCStream does not require us
to manually modify applications' code.

Rho {\it et al.} proposed a stream management technique, called FStream, at the
file system layer~\cite{FStream}. File system's metadata and journal data that
may have different lifetime characteristics were allocated to separate streams.
Moreover, by separating user data based on filenames or extensions, it further
improved the accuracy of data separation. Since FStream was implemented as part
of a file system, it was not able to detect application's I/O behaviors.
Moreover, FStream may be hard to be deployed in practice because of its strong
dependency with file system implementation. Unlike FStream, PCStream is not
only able to detect application-specific behaviors, but also dose not require
any modification of file systems.

Yang {\it et al.} presented the automatic stream management technique at the
block device layer. Similar to existing hot-cold separation techniques, it
decided the lifetime of data based on LBAs where they are written to and
assigned data to different streams depending on their update frequency.
Unfortunately, it does not work well for applications ({\it e.g.}, RocksDB,
Cassandra, and GCC) which have no strong relation between hotness of data and
LBAs.  The proposed PCStream detects the lifetime of data using PCs, so it
works well even for append-only and write-once workloads.

Ha {\it et al.} proposed an idea of using PCs to separate hot data from cold
one in an FTL layer~\cite{ha}. Kim {\it et al.} extended it for multi-streamed
SSDs~\cite{PCStream}.  Our work has improved the previous approaches in a more
complete fashion. First, we showed the importance of the internal stream that
was performed inside an SSD guided by a host-level stream manager. Second, we
showed that PC information could be used as a useful hint for various
applications and workloads, including indirect write applications, append-only,
in-place update, and write-once workloads. Third, we showed that the unique and
consistent nature of PCs made it possible to capture long-term I/O behaviors of
both long-lived and short-lived applications.  Finally, more realistic and
detailed evaluations using various applications and a real SSD were presented
in this paper.


