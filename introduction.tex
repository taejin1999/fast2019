\section{Introduction}
\label{sec:intro}
In flash-based SSDs, garbage collection (GC) is inevitable because NAND flash 
memory does not support in-place updates.  
Since the efficiency of garbage collection significantly affects  
both the performance and lifetime of SSDs, garbage collection has been extensively 
investigated so that the garbage collection overhead can be reduced
~\cite{GCGreedy, GCVictim, GCTTFlash, HotCold}.  
For example, hot-cold separation techniques are commonly used inside an SSD 
so that quickly invalidated pages are not mixed with long-lived data in the same block.   
For more efficient garbage collection, many techniques also exploit
host-level I/O access characteristics which can be used as useful hints on 
the efficient data separation inside the SSD~\cite{JiTGC, ShadowGC}.

Multi-streamed SSDs provide a special interface mechanism for 
a host system, called streams\footnote{In this paper, we use ``streams''
and ``physical streams'' interchangeably.},
with which data separation decisions 
on the host level can be delivered to SSDs~\cite{T10, MultiStream}.  
When the host system assigns two data $D_1$ and $D_2$ to 
different streams $S_1$ and $S_2$, respectively, a multi-streamed SSD 
places $D_1$ and $D_2$ in different blocks, which belong to $S_1$ and $S_2$, respectively.
When $D_1$ and $D_2$ have distinct update patterns, say, $D_1$ with a short lifetime 
and $D_2$ with a long lifetime, allocating $D_1$ and $D_2$ to different streams 
can be helpful in minimizing the copy cost of
garbage collection by separating hot data from cold data.  
When streams are properly managed, they can significantly
improve both the performance and lifetime of 
flash-based SSDs~\cite{MultiStream, Level, FStream, vStream, AutoStream}.
We assume that a multi-streamed SSD supports $n$ streams, $S_0$, ..., $S_{n-1}$.

In order to maximize the potential benefit of multi-streamed
SSDs in practice, several requirements need to be satisfied both for 
stream management and for SSD stream implementation.
First, a fully automatic management of streams is essential without requiring 
any program changes for stream management over general I/O workloads.
For example, if an
application developer should manage stream allocations {\it manually} for 
a given SSD, multi-streamed SSDs are difficult to be 
widely employed in practice.   If the application
developer has no expertise on the application's I/O characteristics or 
the underlying file system, it will be quite challenge to allocate streams efficiently.   
Second, stream management techniques should have no dependency on 
the number of available streams.  
If stream allocation decisions have some dependence on
the number of available streams,  
stream allocation should be modified
whenever the number of streams in an SSD changes.
Third, the number of streams supported in an SSD should be sufficient 
to work well with multiple concurrent I/O workloads.  
For example, with 4 streams, it would be difficult to support a large
number of I/O-intensive concurrent tasks.  

To the best of our knowledge, however, no existing techniques meet all these requirements.  
Most existing techniques~\cite{MultiStream, Level, FStream, vStream} require 
programmers to assign streams at the application level with manual code modifications.
For example,  
in both \textsf{\small vStream}~\cite{vStream} and
\textsf{\small ManualStream}\footnote{For brevity, we denote the 
manual allocation method used in~\cite{MultiStream} by \textsf{\scriptsize ManualStream}.}~\cite{MultiStream}, 
it is an application developer's 
responsibility to assign a given write request to a specific stream.
\textsf{\small AutoStream}~\cite{AutoStream} is the only known automatic technique 
that supports stream management in the kernel level without manual stream allocation.
However, since \textsf{\small AutoStream} predicts data lifetimes using the update frequency 
of the logical block address (LBA), it does not work well with modern append-only workloads 
such as RocksDB~\cite{RocksDB} or Cassandra~\cite{Cassandra}.  
Unlike conventional update workloads where data written to the same LBAs 
often show strong update locality, 
append-only workloads make it impossible to predict data lifetimes 
from LBA characteristics (such as access frequencies or access patterns).  

In this paper, we propose a {\it fully-automatic} stream management technique, 
called \textsf{\small PCStream}, which works efficiently over all the I/O workloads including modern 
append-only workloads.   
The key insight behind \textsf{\small PCStream} is that stream allocation decisions should be made 
at a higher abstraction level than the LBA level where I/O 
activities can be distinguished.
For example, in RocksDB, if we can tell whether the current I/O is a part of 
a logging activity or a flushing activity, stream allocation decisions can be made 
a lot more efficiently over when only LBAs of the current I/O is
available.   
Furthermore, 
our preliminary evaluation results strongly indicate that most 
dominant I/O activities (such as the logging activity in RocksDB) tend to show distinct
data lifetime characteristics.

In \textsf{\small PCStream}, we employ a write program context\footnote{In this paper, since we are 
interested in write-related system calls such as write() in Linux, 
we use {\it write program contexts} and {\it
program contexts} interchangeable where no confusion arises.} as such a higher-level 
classification unit for representing I/O activity regardless of the type of I/O workloads. 
A program context (PC)~\cite{PC, PC2}, which represents a particular execution path of a program 
up to write system calls,
is known to be effective in separating data lifetimes~\cite{PCHa}.  
By identifying dominant I/O activities using program contexts during run time, 
\textsf{\small PCStream} can fully automate the whole process of stream allocation within the 
kernel with no manual work.  In order to
seamlessly support various SSDs with different numbers of streams, \textsf{\small PCStream} first maps 
each program context to a logical (i.e., virtual) stream with no restriction 
on the number of logical streams.
For logical streams allocated, \textsf{\small PCStream} groups ones with similar data lifetimes 
and assigns them to physical streams using the k-means clustering algorithm~\cite{kmeans}. 

By using program contexts as stream allocation units, \textsf{\small PCStream} easily 
supports different I/O workloads regardless of whether it is update-only or append-only.   
Since program contexts focus on the
semantic aspect of I/O execution, not on the low-level details such as 
LBAs and access patterns, \textsf{\small PCStream} does not need to handle different I/O workloads 
separately.  Furthermore,
since program contexts can be uniquely identified over (virtually) all the 
existing applications, \textsf{\small PCStream} can maintain each stream's characteristics 
with each process, thus enabling quick and
accurate stream allocation whenever the same application is executed.

Although many program contexts show that their data lifetimes are 
narrowly distributed, we observed 
that program contexts alone are not sufficient to perfectly distinguish data lifetimes.  
For example, when a database app updates its 
tables under several different execution contexts (i.e., different program contexts),
although we can identify 
different program contexts, their data
lifetime distributions may not be accurately predicted because database 
updates tend to occur in a random fashion.  
Data lifetime distributions of such program contexts may have rather large variances.
In \textsf{\small PCStream}, 
when such a program context {\it $PC_j$} is observed 
(which was mapped to a stream {\it $S_k$}), 
the long-lived data of {\it $PC_j$} are moved to the substream of {\it $S_k$}
during GC.  
The substream prevents the long-lived data of the stream {\it $S_k$} 
from being mixed with future short-lived data of the stream {\it $S_k$}.

When several program contexts have a large variance in their data lifetimes, 
the required number of streams can quickly increase over the number of physical streams.
In order support substreams effectively, we implement a new stream type, 
called internal streams, which can be used for substreams only. 
Unlike physical streams, internal streams can be efficiently
implemented at low cost without increasing the SSD resource budget.  
In the current version of \textsf{\small PCStream}, we create the same number of internal streams 
as the physical streams, effectively doubling the number of available streams. 
By dedicating internal streams only for substreams, \textsf{\small PCStream} can handle 
program contexts with large variances without consuming valuable normal
streams.

In order to evaluate the effectiveness of \textsf{\small PCStream}, 
we have implemented \textsf{\small PCStream}
in the Linux kernel (ver. 4.5) and 
extended a Samsung PM963 SSD to support internal streams.
Our experimental results show that \textsf{\small PCStream}
can reduce the GC overhead as much as a highly-optimized 
manual stream management technique while requiring no code modification.  
Furthermore, \textsf{\small PCStream} outperformed \textsf{\small AutoStream} 
by reducing the average WAF by 35\%.

The rest of this paper is organized as follows. 
In Section 2, we explain the key limitations of 
existing stream management techniques and stream implementation methods.
Before describing \textsf{\small PCStream}, its two core components are 
presented in Sections 3 and 4.
Section 5 describes \textsf{\small PCStream} in detail.
Experimental results follow in Section 6,
and related work is summarized in Section 7.  
Finally, we conclude with a summary and future work in Section 8.

